\subsection{System I/O} \label{sec:sys-io}
The only I/O that will be exposed from the device will be the USB port. Internally, there will also
be a SWD programmer header for ARM programmers and debuggers.

\subsubsection{USB-UART Bridge}
Another design section of Aether's power system is the IC used to translate USB communication data
to UART interface. This is essential because host communication and software implementation will be
done through USB input. However, Lora-E5 will only take UART, therefore a conversion IC is needed.
This design uses Silicon Lab's USBXpress Family CP2102N chip. The Data(D-) and Data(D+) pins on the
bridge are connected to the same pins on the USB and are separated by the protection circuit
previously discussed. The TX and RX pins are connected to the USB serial RX and TX pins on the
micro-controller. LED diodes are set in place to indicate a proper communication link has been
established. The features of the CP210N chip eliminates the need of firmware and driver development
and provides a simple and quick method for USB connectivity.

The next two pins connected on the microcontroller deal with the USB communication interface. As
discussed before, this microcontroller is not equipped with the ability to communicate directly
through USB. Once the USB-to-UART conversion circuit was designed the TX and RX outputs on the
USB-to-UART bridge were connected to 2 GPIO pins on the microcontroller. Pin 10 (PB6) is responsible
for taking USB TX data and Pin Pin 9 (PB7) takes in USB RX output. The TX/RX pins are responsible
for transmitting and receiving USB data. Pin 12 (PC1) on the microcontroller is a GPIO pin dedicated
to the SHDN pin for the boost converter that takes the Li-Io battery as an input and outputs
a consistent 5V for the PM sensor. The microcontroller will be responsible for initiating
a low-power mode state for the nodal system which shuts down the PM sensor remotely as on of the
means to conserve power. Lastly, pin PB3 is an MCU GPIO configured as an EXTI (external interrupt)
source. This pin is responsible for detecting when the USB source is connected and instructs the
microcontroller to wake up. 

\subsubsection{The SWD Programming Header}
The SWD (Serial Wire Debug) will be used for debugging and programming the device during
development, and in the final version of our project. The SWD interface is simpler than JTAG, and
only consists of a ground pin, to match the MCU's ground, serial data (SWDIO), serial clock
(SWCLK), and a reset pin to drive the MCU in reset in the correct sequence to begin flash
programming. The header will use a simple 5-pin 2.54mm male header. Table \ref{tab:swd-header} maps
the LoRa-E5 pins to the STM32WLE5JC pins (internal to the LoRa-E5) \cite{ds-stm32wle5j8}.

\begin{table}[H] 
  \footnotesize  
  \caption{The SWD Header Pin Mappings \cite{ds-stm32wle5j8}\cite{ds-lora-e5}}
  \begin{tabularx}{\linewidth}{| X | X | X |}
    \hline
    Function & LoRa-E5 & STM32WLE5JC 
    \\\hline\hline

    Reset
    & 17
    & RST
    \\\hline

    SWDIO
    & 3
    & PA13
    \\\hline

    SWCLK
    & 4
    & PA14
    \\\hline

    Ground
    & 2
    & GND
    \\\hline

    Reserved
    & Not connected
    & Not connected
    \\\hline

  \end{tabularx}
  \label{tab:swd-header}
    % \caption*{\footnotesize Note: the paper submission date is December 7th, 2021}
\end{table}

