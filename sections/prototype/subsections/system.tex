\subsection{Backend and Webapp Prototype}
As an initial prototype and proof of concept, the backend and webapp will implement very simple
functionalities. The goal of the proof of concept (PoC) is to test simple integration with the
development board, the AWS LNS, the database, and the React webapp. The backend services, initially,
will only be setup to log decoded LoRaWAN packets in the AWS DynamoDB, and display the logs in
a table using the MUI React UI framework. From there, work can be done on integrating the Google
Maps API, and generating custom data overlays using tools available in the API. The differences
between the proof-of-concept and the minimally-viable-concept phase are tabulated in Table
\ref{tab:prototype-phase-comparison}.

\subsubsection{The Proof of Concept}
As aforementioned, the proof of concept will only consist of taking data readings on from the
development board of a light sensor, transmitting it over LoRaWAN to the AWS LNS, decoding the
binary packets and logging them into the AWS DynamoDB, and then displaying the logs in the React
webapp using the MUI framework. There will be no user authorization, or even a concept of a user at
this phase. End devices will be manually configured on the AWS LNS through the AWS IoT Core service.
As tabulated in Table \ref{tab:fall-milestones}, the PoC should be completed before the Spring
semester, but ideally before the start of the new year (2022). The proof of concept will also
include email and SMS notifications when a light threshold is met.

\subsubsection{The Minimally Viable Product}
The Minimally-Viable-Product (MVP) prototype phase implements the absolute basic requirements of the
project, such as the interactive map. Users will be able to register their devices on the webapp,
but not many other features will be available for controlling and configuring devices remotely.
Also, only the interactive map with overlays will be created, and any static or real-time
time-series graphs/overlays won't be covered in this phase. The user will only be able to see the
status of their sensor node and see its exact location (whereas for non-owners, the location won't
be pinpointed). The interactive map will allow the user (doesn't require login) to toggle sensor
readings on the overlay, and hovering over areas of the map will reveal an info window for more
details about the sensor data in that area. The MVP will extend the notification system to tweet on
an Aether Twitter account about levels seen in the network deemed by the EPA as mildly unhealthy or
worse.

\begin{table}[H]
\centering
  \begin{tabularx}{\linewidth}{|l|c|c|}
    \hline
    Feature & Proof of Concept & Minimally Viable Product 
    \\\hline\hline
  
    Logging & \yes & \yes
    \\\hline

    Email and SMS Notifications & \yes & \yes
    \\\hline

    Twitter Tweets as Notifications & \no & \yes
    \\\hline

    AQI Sensor Data & \no & \yes
    \\\hline

    Data Map Overlay & \no & \yes
    \\\hline

    User Registration/Authorization & \no & \yes
    \\\hline

    Time Series Graphs & \no & \yes
    \\\hline

  \end{tabularx}
  \caption{Differences between the two prototype phases}
  \label{tab:prototype-phase-comparison}
\end{table}
