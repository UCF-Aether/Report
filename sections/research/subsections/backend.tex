\subsection{The Backend Software Stack}
The backend software stack will be responsible for collecting data from the
sensor nodes and displaying the air quality data on webpage. It will also be
responsible for sending alerts via text or on Twitter about poor air quality
conditions. It will be the main way users will interact with their deployed
sensors and view data. The backend will also allow users to manage, configure,
and send commands to the sensor nodes remotely. The backend is composed of three
main components: (1) the LoRaWAN gateways running the LoRaWAN Network Server,
(2) a server running on an IoT host provider that can collate data from the
gateways and provide other backend services, and (3) a web user interface
running on the IoT backend provider (such as AWS IoT or Azure IoT) that displays
statistics and user specific information. Additionally, the LNS can optionally
be ran in the cloud. In this scenario, the gateway runs a LoRa packet forwarder
that understands the LNS protocol. Packets can be forwarded to the LNS data
endpoint server, and the gateway is managed through a configuration and update
server (CUPS) using the CUPS protocol along with HTTPS. While the following
software solutions offer support for LoRaWAN versions 1.0.x and 1.1, since the
chosen LoRa module only supports up to 1.0.3, the following discussions will be
with LoRaWAN 1.0.3 as there are some differences with the architecture. The
changes are in regard to the process of join the network. 

There are two types of LoRaWAN networks: public and private. Public networks are
networks akin to Verizon or AT\&T. They provide coverage and networking. Two of
the biggest LoRaWAN public networks are The Things Network and The Helium
Network. Once registered on the network, LoRaWAN devices can operate with the
developer's application server which is configured during registration. The
other option is to configure a private network. When creating a private network,
there are generally two main architectures to choose between in order to route
packets to an application server. The first option, which will be referenced as
local-LNS, is to run the entire LNS on a gateway device. This means that the
gateway needs to manage more information about the end-devices and the state of
the network. The second option, cloud-LNS, is where the gateway operates purely
as an access point, and forwards packets to a LNS in the cloud.  This offers
greater scalability and lower compute and power requirements of the gateway
device at the expense of added complexity. Luckily, there are many solutions
available through service providers, such as AWS.

\subsubsection{LoRa Packet Forwarding}

\subsubsection{LoRaWAN Network Servers} The sensor nodes communicate over LoRa
using the LoRaWAN protocol to gateways.  These gateways are running what is
called the LoRaWAN Network Server (LNS), which is a standardized protocol for
connecting devices using LoRaWAN to the internet.  LoRaWAN network servers are
the bridge between LoRa and the Internet.  LoRa network servers can be public or
private, and end-devices must first be registered on the network before it can
join it. All packets sent by an unregistered and unverified end-device will be
discarded.

As previously mentioned in the LoRaWAN standardization section,
devices communicating over LoRaWAN perform a handshake (joining the network),
and exchange a set of IDs that identify the device, the join server, and the
application/network session. These IDs are: \texttt{DevEUI}, \texttt{JoinEUI} (known previously as
the \texttt{AppEUI}), and \texttt{AppKey}, respectively. The remainder of the keys are generated
from this set of keys. The \texttt{AppKey} is the only key that isn't transmitted over
the network, and is stored on both the end-device and the network server. For
LoRaWAN 1.1+, an additional key is required to be stored on the end-device and
the network server: the \texttt{NwkKey}. The \texttt{NwkKey} is like the \texttt{AppKey} in that it is
never transferred and is used as an extra encryption layer. The network server
will need to be integrated and capable of communication with the application
server, which in for the project, will either be Amazon AWS or Microsoft Azure.

For public networks, which are basically like Verizon or AT\&T for LoRa, the
network server software is provided to users who want to setup a gateway. Once
setup, anyone with reception will be able to use the network. The Things Community Stack
is a free, public network. Users that create a gateway are not financially
incentivized, but only are if they have a end-device they would like to use in
their area. On the other hand, The Helium Network provides financial incentives
without a central authority through a specialized blockchain. For people and
companies that would like to setup a private network, an open-source network
server, called Chirpstack, is available.


\paragraph{The Chirpstack Network Server Stack}

\subsubsection{Public LoRaWAN Networks} \label{public-networks}
Public LoRaWAN networks allow users to connect their own LoRa end-devices to them and begin sending data to the internet. These networks make it easier start using LoRaWAN devices and removing the need to purchase, set up, and maintain large amounts of LoRaWAN gateways. Some public networks, like The Things Community Network, are entirely free to connect to and use. However, there are others, such as The Helium Network, that take advantage of blockchain technology to charge users for the usage of the network. This functions as a way to provide a monetary incentive for users to add gateways to the network, making it larger and increasing it's coverage.

\paragraph{The Things Community Network}
The Things Community Network is a public LoRaWAN network that uses The Things Network Stack. This network was previously known as simply The Things Network. The software for this network is entirely open source, so it would be possible for someone to run their own instance of this network on their own servers. Currently, The Things Network has over 21,000 gateways in operation providing great network coverage. However, the vast majority of network coverage is in Europe. In the United States, coverage is largely centered around the northeast, with some coverage in Florida.

The Things Community Network provides a tool for adding LoRaWAN devices to the network known as The Console. It is a web application which can be used to register applications, end devices or gateways, monitor network traffic, or configure network related options, among other things. When adding a device to the The Things Community Network, there are preset device profiles for mainstream, off-the-shelf devices to make set up very straightforward. Otherwise, if a user is attempting to add a custom device, they will have to manually provide their device information (\texttt{DevEUI}, \texttt{JoinEUI}, etc.). Due to the free and public nature of The Things Community Network, they specifically mention that users should take actions to limit network use, such as limiting data transmission frequency, optimizing message encoding for size, and avoiding confirmed uplink messages \cite{the-things-network}. They recommend these practices to ensure that no single user is hoarding the usage of network resources. These sort of policies and suggestions are necessary when there is no cost to use a network.

\paragraph{The Helium Network}
The Helium Network is a public LoRaWAN network that takes advantage of blockchain and cryptocurrency technology and is made up of so-called Helium Hotspots. Hotspots produce and are compensated in HNT, which is the native cryptocurrency of the Helium blockchain. The goal of the Helium blockchain is to incentivize people to set up LoRaWAN gateways to increase the coverage of the network itself.

The Helium blockchain itself is based off of a concept known as Proof of Coverage. The goal of the Proof of Coverage algorithm is to verify that Hotspots are located where they claim to be located, as well as that they are accurately reporting the area that they claim to cover. The algorithm attempts to continuously verify these details. The Proof of Coverage does this by taking advantage of the unique properties of radio frequencies. Due the fact that the strength of a radio frequency signal is inversely proportional to the square of the distance from the transmitter and the fact that radio frequencies travel at the speed of light with effectively no delay, the Helium network can use these properties to issue challenges to the Hotspots. Completing challenges is used to determine how much HNT a Hotspot is paid.

The developers provide a tool designed to assist users in registering their devices for use on the Helium Network. This tool is known as the Helium Console. Helium requires the creation of a user account. To add a device to Helium, users must either provide the \texttt{DevEUI}, \texttt{AppEUI}, and \texttt{AppKey} that came on the device or use the one auto-generated by the Console. All devices are also provided a unique identifier when connecting to the network. The Helium network supports any LoRaWAN capable device meeting the LoRaWAN v1.0.2 specification. The LoRa module we chose for our design meets this specification.

In order to discuss how to set up your own LoRaWAN gateway as a Helium Hotspot, we must first describe the different classifications of Hotspots that Helium describes. The first type is known as a Full Hotspots. Full Hotspots maintain a full copy of the HNT blochain. This class of Hotspot is able to participate in Proof of Coverage rewards and receive awards for forwarding data packets. However, in order to be classified as a Full Hotspot, the hardware vendor of the gateway must submit an approval form that is approved by the Helium Community and the Decentralized Wireless Alliance (DeWi). The second type of Hotspot is known as a Light Hotspot. Light Hotspots use Validators to get information about the HNT blockchain. This class of Hotspot is able to participate in Proof of Coverage rewards and receive awards for forwarding data packets. The final type of Hotspot is known as a Data Only Hotpsot. These Hotspots can only receive HNT through participating in data packet forwarding. There is no community permission required to add a Data Only Hotspot to the Helium Network. A summary of these Hotspot types can be found in Table \ref{tab:helium-hotspot-classes}.

It is important to note that there is an additional entity that is a part of the Helium Network, but it is not a Hotspot. It is known as a Validator. A consensus group of Validator nodes receive Proof of Coverage and device-related transaction requests from Light Hotspots and both verifies these requests and reaches an agreement on the ordering before forming a new block and adding it to the blockchain.

\begin{table}

\centering
\caption{Summary of Helium Hotspot types}
\begin{tabular}{|l|l|l|l|}
\hline
Rewards Type & Data Only Hotspots & Full Hotspots & Light Hotspots \\
\hline
Network Data Forwarding & Yes & Yes & Yes \\\hline
Proof of Coverage & No & Yes & Yes \\\hline
\end{tabular}
\label{tab:helium-hotspot-classes}
\end{table}

\subsubsection{Application Service Providers}
\paragraph{Microsoft Azure}
Microsoft Azure is a cloud computing service that is operated by Microsoft through their own data centers.

\paragraph{Amazon AWS}

\subsubsection{The Web Application Software Stack}

