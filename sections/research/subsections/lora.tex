\subsection{LoRa Networking and Compute Module} 
Most LoRa modules that have integrated RF solutions already have integrated ARM
cores of some kind. This makes development easier, lowers power consumption, and
allows for a cheaper BOM. Many different types of LoRa modules were
considered. The modules that were most sought after were ones with integrated
U.FL or SMA connectors in order to ease PCB development. However, modules that
did not have these connectors were still considered.

\noindent
\\The main considerations in choosing a LoRa module were the following:
% Remove paragraph skip between line above and the list.
{
  % \setlength{\parskip}{0em}
  \begin{itemize}
    \item Power consumption
    \item Transmit power
    \item Receive sensitivity
    \item Software stack quality
    \item Documentation availability and quality
    \item Digital I/O
    \item Analog peripherals
    \item PCB design complexity
    \item Cost
  \end{itemize}
}

Choosing a module with high enough power consumption and receive sensitivity is
important for increasing the range of the device. The transmit power, receive
sensitivity, signal gains, and signal losses affect the link budget.

\subsubsection{The Seeed LoRa-E5}
The Seeed LoRa-E5 utilizes a STM32WLE5JC microcontroller, which is the first to
integrate both a microcontroller and a +22dBm LoRa radio into one silicon wafer.
The Seeed LoRa-E5 simplifies development with the STM32WLE5JC by embedding the
clocks and main RF front-end circuitry into a single package, obtaining up to
+20.8dBm transmit (TX) output and -136dBm receive sensitivity, and takes care
of startup and shutdown power sequencing. All in a 12mm by 12mm package. This
microcontroller provides the best power consumption of 360nA while in standby,
58$\mu$A while in low power sleep current (LPSleep), and 125$\mu$A while in low
power run current (LPRun, <2 MHz) with a max LoRa transmit current of 110mA at
22dBm. STM also provides a hardware abstraction layer (HAL) for the device, and
middleware, such as FreeRTOS and a FAT filesystem driver, and a LoRaWAN
software stack all through their STM32Cube software package. The STM32WL
utilizes an Arm Cortex M4 32-bit RISC core with 256KB flash memory and 64KB of
SRAM. It also has a 12-bit ADC, which will be needed for the sensing
sub-system.  I/O connectivity is provided through also has 3 UART controllers,
1 SPI controller, 1 I$^2$C controller, programming pins, and 6 to 8 GPIO,
depending on the UART configuration.

However, the module provides no integrated U.FL connector, which increasing the
complexity of the PCB design and increases the number of components in the BOM.
When compared to other modules, this poses as a big drawback to using this
module along with the fact that it provides no USB or Bluetooth connectivity out
of the box. Even then, the module simplifies the RF traces needed outside the
module, and the reference design (development board) uses a short, straight
trace to a SMA connector with a 0$\Omega$ resistor to allow the user to switch
to a U.FL connector.  Another positive aspect is the fact that it is FCC and CE
certified \cite{ds-stm32wle5j8}\cite{ds-lora-e5}.


\subsubsection{The Midatronics Windy} 
The Midatronics Windy also utilizes a STM32WLE series microcontroller
(specifically the STM32WLE5J816), but otherwise is almost identical to the Seeed
LoRa-E5. The only real difference is the Midatronics Windy is a fully integrated
surface mount device module that also includes an U.FL connector on the package.
This drastically decreases the PCB complexity and requirements, but is double
the cost of the Seeed LoRa-E5. The analysis of the various trade-offs between the
PCB requirements (material, trace width, trace spacing, etc) and the LoRa module
chosen is discussed in a later section. Just like the Seeed LoRa-E5, it is FCC
and CE certified. However, not much else is publicly documented, but the rest
can be assumed since it uses almost an identical microcontroller to the Seeed
LoRa-E5 \cite{ds-midatronics-windy}\cite{ds-midatronics-mkr-windy}\cite{ds-stm32wle5j8}.

% TODO: find where the fuck they have more detailed info about power consumption
% Datasheet doesn't say jack shit about power/sleep modes!
\subsubsection{The Microchip WLR089U0}
The Microchip WLR089U0, like the Midatronics Windy, is a fully integrated
surface mount device module with an U.FL connector. The module utilizes an Arm
Cortex M0+ processor with 256KB of flash memory and 40KB of SRAM. It has a
18.6dBm transmit power, and a receive sensitivity of -136dBm. It has ultra-low
power sleep modes that offer low current consumption of as low as 790nA, 12.6mA
while receiving, and 114.6mA while transmitting at 18.5dBm. There is not much
documentation about the various power modes of the device. It does offer a
12-bit ADC with 7 channels which allows for easier sampling of multiple inputs,
albeit at a lower rate, with only one physical ADC. Since an air quality sensor
doesn't need that many readings per second (when active and taking
measurements), the lower sampling rate doesn't pose as an issue. The module also
has I$^2$C, SPI, UART, and USB I/O peripherals. Having a USB I/O digital
interface built into the module saves money and design complexity, and makes up
for the module's lower performance \cite{ds-wlr089u0}\cite{ds-atsamr34j18}.

\subsubsection{RAK 4600}
The RAK 4600 LPWAN module is based off the Nordic nRF52832 MCU and the
Semtech SX1276 Lora transceiver. The Nordic MCU is a general purpose,
multi-protocol MCU capable of Bluetooth 5.2 with a +4 dBm TX output and Near
Field Communication (NFC). The addition of peripherals would increase the
usability and accessibility of the air quality sensor by allowing the user to
connect and configure the sensor wirelessly through their phone or computer. The
Semtech SX1276 empowers the RAK 4600 to have a LoRa TX output of +20 dBm while
using 125mA, and -148 dBm RX sensitivity while using a max of 14mA. However,
only either BLE or LoRa can be used, not both at the same time. The Nordic MCU
supports ultra low-power sleep states than can achieve down to 1.2$\mu$A of draw
current. However, RAK states that that the RAK 4600 module can only get down to
11.5$\mu$A at 2.0V input and 12.5$\mu$A at 3.3V.

The downsides of the RAK 4600 mostly come down to 1) its sleep power
consumption, 2) its complete lack of I/O, especially analog I/O, and 3) its lack
of a LoRaWAN software stack. The RAK module does not allocate any pins for
analog I/O even though the Nordic MCU has a plethora of digital and analog
peripherals connected to its pins. The RAK module surprisingly mostly has no
connect (NC) pins. Not only that, but it only has one set of UART pins and one
set of I$^2$ pins. While the addition of Bluetooth would be a great addition, it
would greatly complicate the sensor sub-system by not providing any analog I/O
whatsoever. While the lack of a LoRaWAN stack is definitely not a deal breaker
(and would be a great learning experience), it is definitely something to take
into account, especially since it seems to be the standard across most LoRa
modules researched \cite{ds-rak4600}.

\subsubsection{The Onethinx Core}
The Onethinx is another module that supports both LoRa and Bluetooth in a small
25mm by 20mm package. The module includes embedded wire trace LoRa and Bluetooth
antennas, greatly simplifying PCB design. The wire LoRa antenna still provides
+20dBm out TX output power, but is lacking compared to the other modules with
its -94dBm RX sensitivity. It is powered by the Cypress PSoC 6x series MCUs,
which houses an ARM M0+ core, M4 core, and a Bluetooth transceiver, along with a
separate chip, the Semtech SX126x, for the LoRa transceiver. The Cypress PSoC
focuses on power efficiency and security. The Cypress PSoC implements the ARM
Platform Security Architecture (PSA) which provides three levels of
hardware-based resource isolation. Since it has two cores, the user code can
execute separately from the LoRaWAN stack so that the user code can never
interfere with the LoRaWAN communication link. On top of that, the LoRaWAN stack
implements a chain of trust, which essentially is a hash chain formed by hashing
the decrypted code segments at various parts of the stack, and comparing the
hash with the public key. However, the software stack is not publicly available,
and there is not much official or online support (StackOverflow, etc), so if
issues were to arise, it would be much more difficult to fix, especially if
deadlines are quickly approaching. Another factor to consider with the Onethinx
Core module is: the availability. The module is not freely purchasable through
major online electronics distributors (Digikey, Mouser, Arrow, etc). It is only
purchasable through email correspondence, which has proven to be fruitless, even
after 5 business days \cite{ds-onethinx}\cite{ds-sx1261}\cite{ds-psoc64bl}.

\subsubsection{LoRa Module Comparison}
In the following discussion, the LoRa modules are compared. Of the listed
considerations at the beginning of the section, energy efficiency, documentation
quality, and cost were the main deciding factors. The reason for prioritizing
those three characteristics were two-fold: time constraints and cost
constraints. Choosing the part with ample amounts of documentation significantly
affects development speed, and consequently allowing for rapid prototyping and
last-minute changes.

As shown in the table below, the modules are summarized in regard to their
computing capabilities and cost. Most of the modules have comparable costs for
the modules, but the development kits for each of them drastically vary. The
Seeed LoRa-E5 is the most cost effective for both the individual modules and
development boards. Compared to the Midatronics Windy, it is essentially the
exact same system with the only difference being that the Midatronics Windy has
an embedded U.FL connector on the package for double the price. The ARM
Cortex-M4 (and Cortex-M0+ for that matter) is more than enough to meet the
computing requirements of the air quality sensor nodes since all they have to do
is 1) periodically wake up 2) send data to a gateway.  It is more of an added
benefit that the Onethinx Core has two cores. However, having two cores, and
requiring both of them for its specialized, secure LoRaWAN stack, increases the
power usage of the device. Another thing that is a nice-to-have (and uses more
power), but not necessarily a need-to-have, are the Bluetooth peripherals on the
Onethinx Core and the RAK4600. However, having Bluetooth would create a better
user experience, but mainly only for initially configuring the device.

\begin{table}[H]
\centering\scriptsize
\caption{LoRa Module General Comparison}
\begin{tabular}{|r|c|c|c|c|c|c|c|c|}
\hline
Module & Core(s) & Core Clock (MHz) & Cost & Dev Board Cost \\ 
\hline\hline

Seeed LoRa-E5       & Cortex-M4   & 48 & \$8  & \$18  \\\hline
Midatronics Windy   & Cortex-M4   & 48 & \$20 & \$65  \\\hline
Microchip WLR089U0  & Cortex-M0+  & 48 & \$15 & \$114 \\\hline
RAK 4600            & Cortex-M4F  & 64 & \$10 & \$18  \\\hline
Onethinx Core\footnotemark
                    & Cortex M0+ \& M4 & 100/150 & \$20 & \$80 \\\hline

\end{tabular}
% \label{tab:networkingRequirements}
\end{table}
\footnotetext{Not readily available at a distributor or official website. Must
email company.}


When researching and comparing the module's electrical characteristics, we
mostly focused on the power efficiency so that we can meet our goal of having
the battery life can last as least one year. In order to meet our range
requirements, we looked for a LoRa tranceiver with a 150dB direct line-of-sight
link budget. The link budget can be calculated with the equation below, but it
is provided in most, if not all, cases. 

% https://tex.stackexchange.com/questions/110746/equations-and-defining-equation-variables
\begin{gather}\label{eq-link-budget}
P_{RX} = P_{TX} + G - L \\
\intertext{where}
\begin{aligned}
& P_{RX}: \text{received power (dBm)} \\
& P_{TX}: \text{transmitted power (dBm)} \\
& G: \text{gains (dB)} \\
& L: \text{losses (dB)} \\
\end{aligned}\notag
\end{gather}

The full link budget equation that accounts for all sources of gain and loss is
shown below:
\begin{gather}\label{eq-link-budget-full}
P_{RX} = P_{TX} + G_{TX} - L_{TX} - L_{FS} - L_M + G_{RX} - L_{RX} \\
\intertext{where}
\begin{aligned}
& P_{RX}: \text{received power (dBm)} \\
& P_{TX}: \text{transmitter output power (dBm)} \\
& G_{TX}: \text{transmitter antenna gain (dBi)} \\
& L_{TX}: \text{transmitter losses (coax, connectors, etc) (dB)} \\
& L_{FS}: \text{path loss, usually free-space loss (dB)} \\
& L_{M}: \text{miscellaneous losses (fading margin, body loss, etc)(dB)} \\
& G_{RX}: \text{receiver antenna gain (dBi)} \\
& L_{RX}: \text{receiver losses (coax, connectors, etc) (dBm)} \\
\end{aligned}\notag
\end{gather}

The following table tabulates the researched module's electrical
Characteristics. ``TX'' and ``RX'' refer to the transmit power and receive
sensitivity, respectively. The sleep current is defined as the lowest possible
sleep state where the core(s) can still be woken up by a timer at 3.3V. The run
current is the current required to operate all cores on the MCU. Even though all
the modules researched can operate at lower voltages, only 3.3V is considered in
the table. I$_{tx}$ and I$_{rx}$ refer to the current required to transmit and
receive using the LoRa radio, respectively.  The RF characteristics are the
maximum possible values, and the conditions, such as spreading factor (SF) and
bandwidth (BW), may vary between modules.  Lastly, V$_{in}$ is the voltage range
accepted by the module where the module's cores are still operable. By using a
lower voltage, the device can operate at a lower power, at the expense of
reduced RF performance, or the inoperability of a peripheral on the MCU. In some
cases, the run current is either the current in TX or RX modes (LoRa transceiver
can't be turned off), indicated by ``RX|TX''.

\begin{table}[H]
\centering\scriptsize
\caption{LoRa Module Electrical Characteristics Comparison}
\begin{tabular}{|r|c|c|c|c|c|c|c|c|}
\hline
Module & TX (dBm) & RX (dBm) & I$_{sleep}$($\mu$A) & I$_{run}$(mA) & I$_{tx}$
(mA) & I$_{rx}$ (mA) &  V$_{in}$ (V) \\
\hline\hline

Seeed LoRa-E5       & +22.0 & -136  & 0.360   & 3.45    & 107 & 4.82  & 1.8 - 3.3 \\\hline
Midatronics Windy   & +22.0 & -148  & 0.360   & 3.45    & 120 & 12    & 1.8 - 3.3 \\\hline
Microchip WLR089U0  & +18.6 & -136  & 1.61    & RX|TX   & 115 & 12.6  & 1.8 - 3.3 \\\hline
RAK 4600            & +20.0 & -148  & 11.2    & RX|TX   & 125 & 17.0  & 2.0 - 3.3 \\\hline
Onethinx Core       & +20.0 & -137  & 1.66    & 3.67    & 118 & 4.20  & 1.8 - 3.3 \\\hline

\end{tabular}
% \label{tab:networkingRequirements}
\end{table}

The next table shows the I/O capabilities of the modules researched. An
integrated LoRa antenna connector would be ideal, but if the performance of the
module is good, then it doesn't matter as much. Also, it was a requirement that
the module have at least one UART, SPI, and I$^2$ port. Note that while the MCUs
and transceivers used may have a set of I/O capabilities, they may not be
exposed to use on the module. Also, the I/O capabilities tabulated are not
mutually exclusive. Meaning, that if one of the UART ports are enabled, then
there may be less GPIOs available, or even SPI, for example, if they are muxxed
on the same port. However, in most cases, using UART, SPI, or I$^2$C only
removes 2-4 GPIO pins. For the Microchip module, it has 4 general I/O ports that
can configured to be either of the aforementioned communication protocols.

\begin{table}[H]
\centering\scriptsize
\caption{LoRa Module I/O Comparison}
\begin{tabular}{|r|c|c|c|c|c|c|c|c|}
\hline
Module & LoRa antenna & BLE? & USB? & ADCs & SPI & I$^2$C & UART & GPIO \\
\hline\hline

Seeed LoRa-E5       & External  & \no  & \no  & 12-bit, 1ch   & 1 & 1 & 3 & 10 \\\hline
Midatronics Windy   & U.FL      & \no  & \no  & 12-bit, 1ch   & 1 & 1 & 1 & 23 \\\hline
Microchip WLR089U0  & U.FL      & \no  & \yes & 12-bit, 7ch   & 4 & 4 & 4 & 27 \\\hline
RAK 4600            & U.FL      & \yes & \no  & None          & 0 & 1 & 1 & 10 \\\hline
Onethinx Core       & PCB trace & \yes & \no  & 12-bit, 1ch   & 1 & 1 & 1 & 8  \\\hline

\end{tabular}
% \label{tab:networkingRequirements}
\end{table}

\subsubsection{The Chosen LoRa Module: The Seeed LoRa-E5}
Feature and cost wise, the Onethinx and the RAK4600 is a clear winner. However,
the RAK4600's power efficiency, specifically its sleep current, is lacking when
compared to the Seeed LoRa-E5 and Midatronics Windy.  Since the sensor node is
going to be spending most of its time sleeping, having the lowest possible
sleep/standby current is going to be one of the most important factors affecting
the battery life. Not only that, but the documentation is spotty and
inconsistent. Lastly, the RAK4600 does not come with a LoRaWAN stack, unlike
every other module.  Writing a LoRaWAN stack from scratch wouldn't be a huge
hurdle, but since writing a LoRaWAN stack isn't the main goal of our project and
we would rather use already well-tested code, we opted to not use the RAK4600.
The Onethinx then would be the next contender, but the development boards are
costly, and the individual modules are not readily available. In order to
purchase them, we would have to email them and negotiate a price assuming they
would sell in super low quantities (<10). We have reached out on many occasions,
with no response even after two full weeks. Due to not having any confidence
that we can easily source the Onethinx Core module, we opted out of using it.

Next, we considered the Microchip WLR089U0. This module has the most flexibility
when it comes to its I/O. It is unmatched in this area when compared to the
modules tabulated above. Almost everything is great about it, like the
integrated U.FL connector and embedded USB peripheral hardware. However, there
is a limitation when it comes to power management: the LoRa module inside of it
can only be configured in receive or transmit mode and cannot be turned off.
This, and also the relatively high sleep current (that still has an operational
clock), is what turned us away this module.

The Seeed LoRa-E5 and the Midatronics Windy are essentially the same exact
module with the only difference being that the Midatronics Windy has an
integrated U.FL connector. It uses almost the exact same LoRa transceiver/MCU IC
as the Seeed LoRa-E5, but is double the price for the modules, and triple the
price for the development boards. Both of these devices have the best sleep
currents and transceiver performance. They are able to go into a standby mode
with a running RTC that is still able to wake up the CPU - all while operating
at 360nA. One downside of using the Seeed LoRa-E5 is that it does not have an
integrated antenna connector. However, all of the RF front-end is embedded
inside the module, and all that is needed outside of it is a very short trace
connecting to a U.FL or SMA connector. Another added benefit is that Seeed and
STM worked closely together to create the module, so it can be assumed that the
module will perform as intended.
