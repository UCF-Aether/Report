\subsection{Software Tools}
For this project, we used a variety of software to assist in different areas of the project. These areas in include software for project management, communication, project development, and documentation. Choosing the correct software tool for the job is critical to ensure that we are able to use our time as productively and efficiently as possible. Without all of these pre-existing tools, we would likely spend far more time on tasks unrelated to working on our design.

\subsubsection{Team Communication and Project Management}
It is important for teams of any nature to stay organized and have structured and efficient ways of communicating with one another. Before we even got started in the design process, we made sure that we had a tool to organize our project, share files, as well to communicate with one another.

\paragraph{Discord}
Discord is a free, web-based, team communication platform. Discord supports multiple voice communication and text channels. We took advantage of these feature by creating multiple text channels that were each centered around different topics, such as parts, hardware, and software discussion. We primarily used this tool for general communication, as well as quick file sharing. Our biweekly team meetings, if we hold them online, are held over Discord. Discord also supports screen-sharing, which was useful when conducting our team meeting as it allowed us to show to each other what we were working on or to help us solve project issues as team. One of the limitations of Discord is its chatroom-style messaging system. While this is useful when having back-and-forth conversations, it is less ideal when wanting to permanently display certain information. In addition to this problem, Discord also has a file-sharing limit of 5 GB. This is where other tools are needed.

\paragraph{Google Drive}
Google Drive is our primary method of file-sharing and storing project documents. Google Drive is a cloud-based, online storage service operated by Google. This allowed us to efficiently organize documents, such as data sheets, white papers, and any other project related documents. This is also where we stored all of our administrative and financial content, such as the project budget and purchase receipts. In comparison to Discord, Google Drive supports a much larger maximum file size of up to 5 TB.

\paragraph{ClickUp}
ClickUp is the tool we are using for project management. ClickUp allows users to manage projects through the creation of tasks. These tasks can be presented in multiple different ways, such as an AGILE style board, Gantt chart, and a calander view. Tasks can also have dependencies. This can help show how certain tasks are related to one another. Each task also has its own page where users can post comments, create subtasks, show task history, and create documentation for the given task. For our project, we have only have one main list of tasks that includes everything we are working on. However, in the future, once we get further into actually designing and building the project, we will likely add more lists for different areas such as hardware and software.

\subsubsection{Project Development}
This section details most of the major programs we used to assist in the development of our project. We did not include supplementary programs, such as text editors or operating systems, as we considered that software to be interchangeable and not specific to this project.

\paragraph{KiCad}
KiCad is a free and open source software suit for electronic design automation (EDA). The software is used for creating circuit schematics and converting them into PCB designs. KiCad supports the importing schematic footprints designed for EAGLE. KiCad also supports a 3D viewer for PCB designs. There are variety of part libraries online that contain parts from major electronics distributors such as DigiKey. SnapEDA, which is website that catalogs millions of electronic components and includes schematic footprints, symbols, and 3D models, also has support for KiCad. KiCad also has comprehensive official documentation. This will make using the software much easier. In addition, there is a large amount of community support, meaning that if we run into any issues that aren't currently documented, we should be able to find assistance. The large community behind KiCad and free and source nature of the program is what lead to our decision to use this software for our schematic generation and PCB creation needs. Another benefit of KiCad is the fact that it can be version-controlled with Git. Due to the way in which KiCad stores its project files, they are easily tracked by Git like any other plain text file. This will make it easier to work on our KiCad PCB design asynchronously with the members of our team and integrate nicely into our existing Git workflow.

\paragraph{STM32CubeIDE}
STM32CubeIDE is the integrated development environment (IDE) created by ST for use with the STM32 family of microcontrollers. We are using one of these such microcontrollers and therefore we are using this software in our development process. The STM32CubeIDE contains a variety of features that simplify the process of writing software for our STM32 microcontroller. The IDE includes a pre-complied ARM C/C++ compiler and debugger. The IDE also includes a library of boilerplate code and templates to quickly get up and running generic functionality for the microcontroller. There is also a graphical debugger. The STM32CubeIDE also contains STM32CubeMonitor, which is a tool that allows us to monitor and diagnose an STM32 application at runtime. Another critical feature of STM32CubeIDE is the STM32CubeProgrammer, the included programming tool which allows us to program the microcontroller over its SWD interface. Due to the fact that STM3DCube IDE was designed for use with our chosen microcontroller, it is clear why we are using this software.

\paragraph{Neovim \& Vim}
Neovim and Vim are terminal-based text editors, and they are used in conjunction with CMake and the
other command line tools in place of the STM32CubeIDE for team members that are comfortable with
doing so, but the STM32CubeIDE will still be used to help generate template code for the MCU's
drivers.

\paragraph{CMake}
CMake is a meta-build tool that's designed to be able to generate Makefiles independent of the
platform being made on. It can also generate custom installers, and deal with specialized compile
targets, such as the chip we're using, the LoRa-E5 which uses the STM32WLE5JC. CMake is used in
conjunction with STM's Cube software package and software tools for a better developer experience.
STM32CubeIDE is based off of Eclipse, and it is slow and buggy. All the tools from generating
binaries, programming the device, and debugging the device can be used with command-line tools
together with open source software. An open source CMake module is being used, called stm32-cmake
\footnote{Found at: https://github.com/ObKo/stm32-cmake}, to link the STM32WL's drivers, HAL,
FreeRTOS, and generate device specific linker scripts.

\paragraph{STLink}
We will be using an open source version of STM's programmer, called "stlink"
\footnote{Found at: https://github.com/stlink-org/stlink} and debugger GUI software that is usable
from the command-line, and integrates well with CMake, writing automation scripts, and, of course,
Vim. stlink provides debugging and programming capabilities. It also is able to trace code
execution, and provide detail information about the debugger and MCU state.

\paragraph{Git}
Git is a free and open source, distributed version control software. It used by major free and open source software projects and by software companies alike. Git allows us to work on projects asynchronously. Git operates on a series of changes, known as "commits", that make up the entirety of a project, which is contained in a repository (commonly referred to as a repo). A commit is simply a change to a single file or multiple files. Each commit is hashed, which gives it a unique value. New commits that a user makes locally can "pushed" to a remote repository so others can view them. Conversely, new commits from other users can be "pulled" back to a local user's machine. A major appeal of Git is being able to go back in time to an earlier commit, if the need arises. 

Another useful feature of Git is the ability create branches. One can think of a branch as simply a side-version of the master repository. Branches are usually used when testing new features or iterating upon the current stable version of a given software. Eventually, when the branch being worked on is finished, it can be merged back into the master branch. For our project, we utilized Git to control our source code for the remote server and the sensor node. We also used it to manage the source code for this report, which was written in LaTeX.

\paragraph{GitHub}
As mentioned in the previous section, Git requires a way for users to access the same repository in order to collaboratively work on it. GitHub provides the solution to this problem by hosting millions of Git repositories online, for free. We decided to use GitHub to host our Git repositories for this project because of the fact that it is free and the fact that all of our group members have use it before and are familiar with it. Overall, GitHub allows for easy and free hosting for all of our project's repositories.

\subsubsection{Documentation}
Before beginning our project, we realized that documenting our work would be critical to our success. This meant that it was necessary to properly plan and organize how we would write our documentation. Therefore, we found it necessary to move away from traditional word processors to a document preparation system that would make collaboration easier as well as make writing very large documents a simpler overall task. This made LaTeX and Overleaf the obvious choices.

\paragraph{LaTeX}
LaTeX is a document preparation system. Traditional word processors, like Microsoft Word, are referred to as What You See Is What You Get (WYSIWYG), meaning that as you are typing the document you see exactly what the final product will look like. This is in contrast to LaTeX, where you write the document using a markup language in plain text and you only see the final document when you compile the source code. This is why the goal of LaTeX is to allow the user to focus solely on the content and not the formatting. This ends up make the appearance of LaTeX document far more consistent in appearance. LaTeX is also ideal for the typesetting of complex mathematics, which is why it often used in technical papers and documentation. Another reason we decided upon LaTeX for our use in our writing Senior Design reports because it automatically keeps tracking of the numbering of tables figures, ensuring that you never accidental refer to the wrong table or figure from within the text. 

Another key feature of LaTeX that factored into our decision to use it was the ease of integration with Git. Since LaTeX operates using plain text files and not binaries like Microsoft Word, this allows us to use the version control features of Git to manage the different versions of our Senior Design reports and to work asynchronously within our team. LaTeX also supports automatic bibliography formatting and styling, making it easy to switch between different formatting, such as APA or IEEE, as well as easily adding or removing new sources. Overall, due to the size, complexity, and the need to collaborate on the same document with our entire pushed us towards the use of LaTeX for our documentation needs.

\paragraph{Overleaf}
Overleaf is an online LaTeX integrated development environment (IDE). Overleaf also has a comprehensive library of user-friendly LaTeX documentation. While LaTeX can be installed locally on all operating systems, we decided to use Overleaf to make it easier for some members of our team. Some members of the team had not used LaTeX previously and were not as comfortable with Git, so to make it easier for them we decided on also incorporating Overleaf into our documentation workflow. This allowed them to not have to worry about the sometimes challenging nature of working with the numerous LaTeX packages or getting the correct version of LaTeX installed on their computers and instead allowed them to immediately get into writing our project documentation using LaTeX.

We also were able to integrate our Overleaf project with our Git repository, allowing the members of our group who preferred using their own local LaTeX environment to pull any new changes from the Overleaf document. This made for seamless collaboration between those on our team using Overleaf and those using a local environment.


