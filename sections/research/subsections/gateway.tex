\subsection{LoRaWAN Gateway}
In a LoRaWAN network, the gateway exists to receive data packets from LoRa end-devices and send those packets to an network server. It is a critical component for our sensor node to function. For our project, the sensor node is the main design focus and where the majority of our time will be spent. Therefore, we opted to explore two non-custom options for adding a gateway to our system. One of these options is to connect to our sensor nodes to an existing network of gateways. The two major networks will explore is The Things Network and the Helium Network. We discuss these two networks in greater detail in section \ref{public-networks}. The second option, which we will discuss in this section, is to create a gateway using a Raspberry Pi and RAK2245 Raspberry Pi Hat, which is an adapter board that provides the Raspberry Pi with the ability to function as a LoRaWAN gateway.

While it is possible that we could function without our own gateway, we did not want to have to rely on public networks to be able to test our sensor node design. We determined that having our own gateway would also make testing and debugging our design easier. There were a variety of factors that lead to us choosing to go with the Raspberry Pi/RAK2245 solution for our gateway. One of the primary drivers were cost. A member of our team already owned a Raspberry Pi and has experience working with one. This helped reduce cost for the project.

In addition to cost, ease of use was another driving factor in our decision. One of the application service providers we are leaning towards using, AWS, already has comprehensive documentation, as well as provides detailed tutorials for setting up a LoRaWAN gateway using the Raspberry Pi/RAK2245 combination as described here. Since the gateway is not part of the engineering design focus of our project, we determined that this method for creating a gateway would help reduce the time spent here and allow us to spend more time on the sensor node and application layer software.