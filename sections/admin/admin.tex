\section{Administrative}
In this section, we discuss content related to the planning of our project. This will include items such as budgeting and project milestones. These items are key to any successful project to ensure that the design stays within the desired cost range and that the project is completed in time.

\subsection{Estimated Budget}
Here we display the budget for each of the major components of our project including the sensor node, one LoRaWAN gateway, and costs related to the server. Since the sensor node is the main design aspect of our project, it is composed of a custom PCB with a multitude of selected ICs and gas sensors. The budget for this sensor node is shown in Table \ref{tab:sensor-budget}. This is in contrast to the LoRaWAN gateway, which we decided we would not be designing. For this component, we simply used a Raspberry Pi, which one of our group members already owned, and a RAK 4600. The budget for the gateway can be found in Table \ref{tab:gateway-budget}. Similarly, we decided that we would be building and hosting our own server cluster and would instead be using Amazon Web Services under the free pricing tier.

\begin{table}[H]
\centering
    \begin{tabular}{|r|c|c|l|}
        \hline
        Component & Unit Cost & Quantity & Total Cost \\
        \hline\hline
        Solar Panel                 & \$30.00 & 1 & \$30.00 \\
        Solar Charge Regulator      & \$5.00 & 1 & \$5.00 \\
        Battery                     & \$10.00 & 1 & \$10.00 \\
        \hline
        LoRa Module                 & \$9.90  & 1 & \$3.00  \\
        PCB                         & \$10.00 & 1 & \$10.00 \\
        \hline
        PM Sensor                   & \$53.83 & 1 & \$53.83\\
        Renesas ZMOD 4510           & \$7.77  & 1 & \$7.77 \\
        Bosch BME688                & \$20.50 & 1 & \$20.50\\
        \hline\hline
        \multicolumn{2}{|}{Multirow} & Total & \$147.00 \\
        \hline
    \end{tabular}
    \caption{The budget for a single sensor node.}
    \label{tab:sensor-budget}
\end{table}

\begin{table}[H]
\centering
    \begin{tabular}{|c|c|c|c|}
        \hline
        Component & Approx. Unit Cost & Quantity & Total Cost \\
        \hline\hline
        Raspberry Pi 4              & \textit{free} & 1 & \$0.00   \\
        RAK 4600       & \$120.00 & 1 & \$120.00 \\
        \hline\hline
        \multicolumn{2}{|}{Multirow} & Total & \$120.00 \\
        \hline
    \end{tabular}
    \caption{The budget for the gateway.} 
    \label{tab:gateway-budget}
\end{table}

\subsection{Milestones}
In Table \ref{fall-milestones} and Table \ref{spring-milestones} we describe our current milestones for the fall and spring semesters respectively. The full descriptions for the fall milestones can be found in Table \ref{fall-milestone-descriptions} and the full descriptions for the spring milestones can be found in Table \ref{spring-milestone-descriptions}. The goal of creating these milestones is to ensure that we stay on track as the semester progresses and that we do not fall behind. As is often the case, we will likely need to add new milestones as new challenges arise.
\subsubsection{Fall Semester}

\begin{table}[H]
    
    \begin{tabularx}{\linewidth}{|c|X|X|}
        \hline
        Date & Milestone & Deliverables \\
        \hline\hline
        Sep. 2nd, 2021 
        & Sensing mechanism finalized 
        & Sensor requirements document \\
        
        \hline
        Oct. 1st, 2021
        & MCU and LoRa module finalized
        & Order confirmation and, if available, shipment information. \\
        
        \hline
        Oct. 8th, 2021 
        & Sensors finalized 
        & Order confirmation and, if available, shipment information. \\
        
        \hline
        Nov. 5th, 2021 
        & Software alpha 1.0 
        & Software alpha 1.0 node and base-station release binaries. \\ 
        
        \hline
        Nov. 19th, 2021 & Prototype v1.0 PCB 
        & PCB masks, fab order confirmation and, if available, shipment information \\
        
        \hline
        Dec 3rd, 2021 & Prototype v1.5 PCB 
        & PCB masks, fab order confirmation and, if available, shipment information \\
        
        \hline
        Dec 10th, 2021 
        & Software alpha 1.5 
        & Software alpha 1.5 node and base-station release binaries. \\ 
        
        \hline
    \end{tabularx}
    \caption{Fall milestones}
    \label{fall-milestones}
\end{table}

\begin{table}[H]
    \centering
    \begin{tabularx}{\linewidth}{|c|X|}
        \hline
        Project Milestone & Description 
        \\
        \hline\hline
        Sensing mechanism finalized &
        Finish research and create design requirements for accurately detecting wildfires in a cost effective way.
        \\
        
        
        \hline
        MCU and LoRa module finalized &
        The MCU, LoRa module/daughter-boards, and Raspberry Pi's are chosen, and multiple developer boards have been ordered to design and test mesh networking.
        \\
        
        \hline
        Sensors finalized 
        & The required sensors are chosen, and multiple parts/developer-boards have been ordered.
        \\
        
        \hline
        Software alpha 1.0 
        & The nodes can communicate and route packets to the gateway, and the gateway can communicate with all or specific nodes on the network. There also exists some sort of command line interface on the gateway to talk to the nodes, but there are no required internet capabilities of the base, yet. The nodes can go to sleep, but a sleep routine isn't configured optimally.
        \\
        
        \hline
        Prototype v1.0 PCB 
        & A PCB first version prototype PCB has been designed and ordered from a fast-turnaround PCB fab containing all required functionalities: power subsystem, MCU, LoRa module connectivity, battery and solar I/O, and sensor I/O with adequate noise isolation and filtering.
        \\
        
        \hline
        Prototype v1.5 PCB
        & PCB prototype with, if any, major issues fixed. Should also include sensor daughter boards.
        \\
        
        \hline
        Software alpha 1.5
        & The software works on the prototype PCB
        \\
        
        \hline
    \end{tabularx}
    \caption{Fall milestone descriptions}
    \label{fall-milestone-descriptions}
\end{table}

\subsubsection{Spring Semester}
\begin{table}[H]
    \begin{tabularx}{\linewidth}{|X|X|X|}
        \hline
        Date & Milestone & Deliverables \\
        \hline\hline
        Jan. 19th, 2022 
        & Prototype v2.0 PCB 
        & PCB masks, fab order confirmation and, if available, shipment information \\
        
        \hline
        Jan. 19th, 2022
        & Enclosure prototype
        & Model and 3D printed enclosure
        \\
        
        \hline
        Jan. 29th, 2022 
        & Software alpha 2.0 
        & Software alpha 2.0 node and base-station release binaries
        \\
        
        \hline
        Feb. 18th, 2022
        & REST API created
        & REST API documentation and testing/validation scripts
        \\
        
        \hline
        Feb. 18th, 2022
        & Enclosure finalized
        & Model and 3D printed enclosure
        \\
        
        \hline
        Feb. 25th, 2022
        & Final v1.0 PCB
        & PCB masks, fab order confirmation and, if available, shipment information 
        \\
        
        \hline
        Mar. 1st, 2022
        & Software release v1.0
        & Software v1.0 node and base-station release binaries. 
        \\ 
        
        \hline
        Mar. 18st, 2022
        & Software release v1.1
        & Software v1.1 node and base-station release binaries. 
        \\ 
        
        \hline
        Mar. 18st, 2022
        & Final v1.1 PCB
        & PCB masks, fab order confirmation and, if available, shipment information 
        \\
        
        \hline
        Mar. 25th, 2022
        & Simple website to visualize results
        & Website front-end and back-end files, and website URL or IP
        \\
        
        \hline
    \end{tabularx}
    \caption{Spring milestones}
    \label{spring-milestones}
\end{table}



\begin{table}[H]    
    \centering
    \begin{tabularx}{\linewidth}{|c|X|}
        \hline
        Project Milestone & Description 
        \\
        \hline\hline
        Prototype v2.0 PCB 
        & This version should fix most or all remaining hardware bugs, and ensure accurate sensor readings. It should also be mountable in an enclosure.
        \\
        
        \hline
        Enclosure prototype
        & An enclosure O-rings, or some other kind of sealant, to keep out dust and water splashes. It should be able to mount the sensors to get environmental readings, and provide space for top-mounted solar-panel.
        \\
        
        \hline
        Software alpha 2.0 
        & The gateway should be able to set the node's sleep routine settings, and schedule any specific times to wake up at. The command-line utility should be robust enough with minimal bugs.
        \\
        
        \hline
        REST API created
        & The gateway should run a server and/or relay to bigger server to provide a REST API for clients to query data about environmental readings pertaining to wildfires.
        \\
        
        \hline
        Enclosure finalized
        & The enclosure that can house a PCB and mostly not get wet should be good enough.
        \\
        
        \hline
        Final v1.0 PCB
        & A PCB should be bug-free, and sent to a proper fab to have high-quality board material and a water resistant coating.
        \\
        
        \hline
        Software release v1.0
        & The software should be able to handle errors gracefully, reliably establish connections on the mesh network, and reliably wake-up and transmit data.
        \\ 
        
        \hline
        Software release v1.1
        & Bug fixes and minor improvements.
        \\ 
        
        \hline
        Final v1.1 PCB
        & Only if needed. Bug fixes and minor improvements.
        \\
        
        \hline
        Simple website to visualize results
        & The website doesn't need to look pretty, but it should provide a nice visualization of the nodes deployed plus data readings over time displayed in a graph.
        \\
        
        \hline
    \end{tabularx}
    \caption{Spring milestone descriptions}
    \label{spring-milestone-descriptions}
\end{table}

