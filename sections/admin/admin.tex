\section{Administrative}
In this section, we discuss content related to the planning of our project. This will include items such as budgeting and project milestones. These items are key to any successful project to ensure that the design stays within the desired cost range and that the project is completed in time.

\subsection{Estimated Budget}
Here we display the budget for each of the major components of our project including the sensor node, one LoRaWAN gateway, and costs related to the server. Since the sensor node is the main design aspect of our project, it is composed of a custom PCB with a multitude of selected ICs and gas sensors. The budget for this sensor node is shown in Table \ref{tab:sensor-budget}. This is in contrast to the LoRaWAN gateway, which we decided we would not be designing. For this component, we simply used a Raspberry Pi, which one of our group members already owned, and a RAK 4600. The budget for the gateway can be found in Table \ref{tab:gateway-budget}. Similarly, we decided that we would not be building and hosting our own server cluster and would instead be using Amazon Web Services under the free pricing tier.

\begin{table}[H]
\centering
    \begin{tabular}{|r|c|c|l|}
        \hline
        Component & Unit Cost & Quantity & Total Cost \\
        \hline\hline
        P105                        & \$35.00 & 1 & \$35.00 \\
        MCP73871                    & \$5.00 & 1 & \$5.00 \\
        PRT-13856                   & \$25.00 & 1 & \$25.00 \\
        \hline
        LoRa-E5                     & \$9.90  & 1 & \$3.00  \\
        PCB                         & \$10.00 & 1 & \$10.00 \\
        \hline
        PM SPS30 Sensor             & \$53.83 & 1 & \$53.83\\
        Renesas ZMOD 4510           & \$7.77  & 1 & \$7.77 \\
        Bosch BME688                & \$20.50 & 1 & \$20.50\\
        \hline\hline
        \multicolumn{2}{|}{Multirow} & Total & \$167.00 \\
        \hline
    \end{tabular}
    \caption{The budget for a single sensor node.}
    \label{tab:sensor-budget}
\end{table}

\begin{table}[H]
\centering
    \begin{tabular}{|c|c|c|c|}
        \hline
        Component & Approx. Unit Cost & Quantity & Total Cost \\
        \hline\hline
        Raspberry Pi 4              & \textit{free} & 1 & \$0.00   \\
        RAK 4600       & \$120.00 & 1 & \$120.00 \\
        \hline\hline
        \multicolumn{2}{|}{Multirow} & Total & \$120.00 \\
        \hline
    \end{tabular}
    \caption{The budget for the gateway.} 
    \label{tab:gateway-budget}
\end{table}

\subsection{Milestones}
In Table \ref{tab:fall-milestones} and Table \ref{tab:spring-milestones} we describe our current milestones
for the fall and spring semesters respectively. The full descriptions for the fall milestones can be
found in Table \ref{tab:fall-milestone-descriptions} and the full descriptions for the spring milestones
can be found in Table \ref{tab:spring-milestone-descriptions}. The goal of creating these milestones is
to ensure that we stay on track as the semester progresses and that we do not fall behind. As is
often the case, we will likely need to add new milestones as new challenges arise.

\subsubsection{Fall Semester}
\begin{table}[H]
    \footnotesize  
    \caption{Fall Milestones}

    \begin{tabularx}{\linewidth}{| c | p{0.25\linewidth} | c | c | X |}
      \hline
      No. & Milestone & Date & Achieved & Deliverables 
      \\\hline\hline

      F.0 
      & Sensing mechanism finalized 
      & Sep. 2nd, 2021 
      & Sep. 14th, 2021
      & Sensor requirements document 
      \\\hline

      F.1 
      & MCU and LoRa module finalized
      & Oct. 1st, 2021 
      & Oct. 1st, 2021 
      & Order confirmation and, if available, shipment information. 
      \\\hline

      F.2 
      & Sensors finalized 
      & Oct. 8th, 2021 
      & Oct. 27, 2021
      & Order confirmation and, if available, shipment information. 
      \\\hline

      F.3 
      & Prototype v1.0 PCB 
      & Jan. 4th, 2022 
      & -
      & Gerber files, part and PCB order confirmations
      \\\hline

      F.4 
      & Software alpha 1.0 
      & Jan. 4th, 2022 
      & -
      & Node and base-station release binaries, React app, AWS backend code zip package 
      \\\hline

    \end{tabularx}
    \caption*{\footnotesize Note: the paper submission date is December 7th, 2021}
    \label{tab:fall-milestones}
    % \caption*{\footnotesize Note: the paper submission date is December 7th, 2021}
\end{table}

\begin{table}[H]
    \centering
    \footnotesize
    \caption{Fall Milestone Descriptions}

    \begin{tabularx}{\linewidth}{| c | p{0.25\linewidth} | X |}
      \hline
      No. & Milestone & Description 
      \\\hline\hline

      F.0 
      & Sensing mechanism finalized 
      & Research on how to properly measure the air quality has completed
      \\\hline

      F.1 
      & MCU and LoRa module finalized
      & The computing and radio IC(s) have been researched and one (or many) have been chosen
      \\\hline

      F.2 
      & Sensors finalized 
      & The sensors needed to implement the sensing mechanism in F.0 have been researched and picked
      \\\hline

      F.3 
      & Prototype v1.0 PCB 
      & The first fully-featured prototype PCB is complete and ready to be sent to the fab
      \\\hline

      F.4 
      & Software alpha 1.0 
      & A version of firmware that is fully capable of reading sensor data and transmitting them
      over LoRaWAN to the AWS LoRaWAN network server. The AWS backend should be capable of collating
      the data and at least be able to tabulate the data. Displaying a map overlay would be a plus.
      \\\hline

    \end{tabularx}
    \label{tab:fall-milestone-descriptions}
\end{table}

%%%%%%%%%%%%%%%%%%%%%%%%%%%%%%%%%%%%%%%%%
%
% SPRING MILESTONES
% 
% Notes:
%   PCB 1.1
%   enclosure (prototype and final)
%   MVP webapp and backend
%   integrated setup with Helium and TTN
%   CLI or TUI interface for the device (USB)
%   stretch website goals
%   stretch firmware goals (operate wired through USB?)
%   PCB 1.5 (final)
%%%%%%%%%%%%%%%%%%%%%%%%%%%%%%%%%%%%%%%%%

\subsubsection{Spring Semester}
\begin{table}[H]
    \footnotesize  
    \caption{Spring Milestones}

    \begin{tabularx}{\linewidth}{| c | p{0.25\linewidth} | c | X |}
      \hline
      No. & Milestone & Date & Deliverables 
      \\\hline\hline

      S.0 
      & Model and 3D print prototype enclosure
      & Jan. 30, 2022
      & 3D model
      \\\hline

      S.2 
      & MVP Firmware
      & Feb. 13th, 2022 
      & Gerber files, part and PCB order confirmations
      \\\hline

      S.1 
      & MVP Webapp and supporting backend infrastructure
      & Feb. 13th, 2022 
      & React app, AWS backend zip package
      \\\hline

      S.3 
      & PCB Revision 1.1
      & Feb. 13th, 2022 
      & Gerber files, part and PCB order confirmations
      \\\hline

      S.4 
      & Software alpha 1.1 
      & Mar. 6th, 2022 
      & Node and base-station release binaries, React app, AWS backend code zip package 
      \\\hline

      S.5
      & Finalized PCB
      & Mar. 13th, 2022 
      & Gerber files, part and PCB order confirmations
      \\\hline

      S.6 
      & Stretch goal completion
      & Apr. 1st, 2022
      & Node and base-station release binaries, React app, AWS backend code zip package 
      \\\hline

      S.7 
      & Finalized design
      & Apr. 11st, 2022
      & Code binaries, 3D model, Gerber files
      \\\hline

    \end{tabularx}
    \label{tab:spring-milestones}
\end{table}

\begin{table}[H]
    \centering
    \footnotesize
    \caption{Spring Milestone Descriptions}

    \begin{tabularx}{\linewidth}{| c | p{0.25\linewidth} | X |}
      \hline
      No. & Milestone & Description 
      \\\hline\hline

      S.0 
      & Model and 3D print prototype enclosure
      & A 3D model has been designed to house the PCB, and has been printed
      \\\hline

      S.2 
      & MVP Firmware
      & The firmware will be able to take sensor measurements from all sensors and upload them to
      the AWS backend over LoRaWAN
      \\\hline

      S.1 
      & MVP Webapp and supporting backend infrastructure
      & The webapp will be able to allow users to view the map with sensor data overlays. Users can
      sign up to register their devices with the AWS LNS
      \\\hline

      S.3 
      & PCB Revision 1.1
      & Fixes to the PCB design, and new prototype design has been ordered
      \\\hline

      S.4 
      & Software alpha 1.1 
      & Any bug fixes and stabilization, and a CLI and/or TUI Linux program has been created to
      interact with the device over USB
      \\\hline

      S.5
      & Finalized PCB
      & Any remaining fixes and final touches to the PCB, and ordered with high quality materials
      and production fab
      \\\hline

      S.6 
      & Stretch goal completion
      & Complete as many stretch goals as possible. Ideally, integration with The Things Network and
      Helium would be nice, along with additional webapp features, such as: controlling devices
      remotely, viewing device logs, and view time-series graphs and map overlays of sensor data
      \\\hline

      S.7 
      & Finalized design
      & Any final bug fixes to firmware and software, along with a finalized enclosure design with
      pivot-able solar panel fixture. All requirement specifications should have been met, along
      with as many stretch goals as possible, such as wired (USB) data uploads
      \\\hline

    \end{tabularx}
    \label{tab:spring-milestone-descriptions}
\end{table}
