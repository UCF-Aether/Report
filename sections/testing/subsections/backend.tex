\subsection{The Backend Testing Plan}
Since both the backend AWS services and web application will be written in
JavaScript/TypeScript (NodeJS runtime for the server-side things), the testing
software tools can be a unified set, simplifying development and testing. Like
with the firmware testing, the backend and web application will have a simple
continuous integration pipeline configured on GitHub to automatically test the
code before merging changes into the main, or (hopefully) stable, branch on the
repository. We will be utilizing the Jest unit testing framework since it has
tight integration with Typescript, Babel, and React. It is also one of the most
popular framework, and is widely used by big companies, such as Spotify,
Facebook, Twitter, and Instagram. Jest was also chosen because it is easy to
use with minimal or no configuration, it's well documented, works with
asynchronous code, and provides code coverage statistics. It'll be imperative
that our unit tests hit the following systems:

\begin{itemize}
  \item Sending and receiving commands and data to and from air quality nodes
  \item Managing information regarding a specific user, their information, and
    configuration in the database and other systems that might touch the data
  \item Registering new devices with the AWS LNS
  \item API endpoints regarding geographical and AQI map information
  \item API endpoints regarding user registration, configuration, and updating
    and retrieving data
\end{itemize}

Integration tests would be nice, but due to time constraints, they might be very
limited. The same is true for unit tests, but a balance will be found if time
would be saved by having the test. Integration tests that might be nice to have
would be:

\begin{itemize}
  \item Registering new devices with the Helium Network or The Things Network
  \item Using map data from Google Maps or another map service provider
\end{itemize}

\subsubsection{Jest}
Each test in Jest is created with the \code{test(name, fn, timeout?)}. Inside the \code{test()}
function, the test is written. To check and verify values, the \code{expect(value)} is used. The
\code{expect()} function returns an \code{Expectation} object, with provides an interface to
\emph{match} the \code{value} with various conditionals, called ``matchers''. For example, you could
have a function that computes a result, and the function is operating correctly if the resultant
value is between 0 and 100. Like, Unity, there is a huge set of matchers available to use. The
\code{.toEqual(value)} matcher would be used, like in listing \ref{lst:jest-example}. Jest provides
the following report after running all of the tests, as shown in listing \ref{lst:jest-output}


\subsubsection{Testing the Serverless Application}
Since we are opting for a serverless backend (using AWS Lambda), the Amazon Serverless Application
Model CLI will be utilized for testing (and for deployment). Through YAML configuration files,
a serverless application can be deployed with the right configuration automatically. It also
provides a local testing environment for Lambda. The AWS SAM CLI also provides debugging
capabilities that allow the developer to step through the code and view information about the
current state of the program. For AWS Lambda, the main functionality will be decoupled as much as
possible from the Lambda handler function to maximize the amount of testing that can be done locally
and through the GitHub continuous integration pipeline without requiring access to AWS services.
