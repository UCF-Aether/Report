\subsection{The Hardware Test Plan}
\subsubsection{Microcontroller Testing}
\subsubsection{Sensor Testing}
When it comes to testing the sensors we must be very thorough and diligent. While every portion of the system is important this is what every other sub system is designed around and for. Should errors persist within any sensor the entire network could be compromised with inaccurate and inconsistent information. The sensors have different needs and functionalities so as a result testing for this system needs to be unique for each and every component.
\paragraph{Testing the SPS30 sensor}
When it comes to testing this sensor we must take several factors into account. Fist of all every subsystem must be tested before this. If it turns out the power subsystem is providing too much or too little power then the sensor could be damaged or not turn on at all. If the microcontroller is damaged then it may not interpret the results correctly. If the communication system is faulty then we may receive incorrect, unreadable, or no response at all. Once we've verified that every other subsystem is correct then and only then is it proper to proceed in testing any sensor.
For a good rule of thumb one should access and make sure that all the wiring's are correct before proceeding with testing the SPS30, one should also take advantage of the convenient testing point on the sensor side of the power rail to ensure that the input voltage is the correct five volts. To properly test the SPS30 a one cubic meter containment field must be constructed that is airtight. From there exactly thirty five grams of finely ground sawdust must be introduced with aid of a fan to thoroughly mix the particulates throughout the containment. The SPS30 will then be powered on to observe the readings and find the accuracy.
\paragraph{Testing the ZMOD4510 sensor}
When it comes to testing this sensor we must take several factors into account. Fist of all every subsystem must be tested before this. If it turns out the power subsystem is providing too much or too little power then the sensor could be damaged or not turn on at all. If the microcontroller is damaged then it may not interpret the results correctly. If the communication system is faulty then we may receive incorrect, unreadable, or no response at all. Once we've verified that every other subsystem is correct then and only then is it proper to proceed in testing any sensor.

\paragraph{Testing the BME688 sensor}
When it comes to testing this sensor we must take several factors into account. Fist of all every subsystem must be tested before this. If it turns out the power subsystem is providing too much or too little power then the sensor could be damaged or not turn on at all. If the microcontroller is damaged then it may not interpret the results correctly. If the communication system is faulty then we may receive incorrect, unreadable, or no response at all. Once we've verified that every other subsystem is correct then and only then is it proper to proceed in testing any sensor.

\subsubsection{Power System Testing}
The essential elements of testing our power sub system can be described as basic and comprehensive. If any single element were to introduce a factor to limit or boost power unexpectedly certain components may not function or become damaged beyond repair. As a result the upmost care must be taken into account when testing the power subsystems of the device. The most effective method of testing in our opinion is to monitor the voltage and current starting at the power sources going all the way to the load. 

\paragraph{Source Testing}
The first step we will undergo would be to test the solar panel. This will be done by using the panel and a voltmeter, one would take the solar panel out during both the most and least ideal conditions. This would mean taking it out on both a sunny and overcast day and monitoring the output over the course over an eight to ten-hour period. This will provide us with an accurate and predictable range of our primary source. This will also alert us of any anomalies that could be present weather that be alarmingly low or high input voltages. Once we have complied information from this test and it matches the expected input we can carry on and test further into the subsystem.
The next step to testing our sources would be to test the USB input. This is a far bit simpler as the USB input would have a standard input we could compare it to and any difference would indicate an inconsistent or unreliable element. Additionally we don't have to worry about a fluctuation in power as the USB input would stay steady at all times and any fluctuation would also indicate an undesirable element. Once again using a voltmeter once the USB charger has been powered on we can monitor the connection several times over the course of an hour to check its consistency. For ease of use testing points have been added at select points that can be used, these points can be found in the schematic and are the points we will be using to test this subsystem. Once we've verified that the USB has expected and consistent power we can carry on testing further into the power subsystem.

\paragraph{IC Testing}
When we get to the IC a new approach must be made. First When monitoring the input from the sources care must be made to monitor the readings directly at the input of the IC as well as the output reading directly from the sources. This will tell us if there is a factor manipulating the power before it reaches its destination. The next step would be to observe the ICs output voltage. Now due to the nature of the system the input voltage of the IC may vary but the same cannot be said about the output voltage. The next step would be to observe the output voltage of the IC as the solar panels input voltage changes. Should the output voltage change in sizeable way then the IC is not accomplishing one of its core purposes and should be examined and repaired or replaced. For ease of use testing points have been added at select points that can be used, these points can be found in the schematic and are the points we will be using to test this subsystem. The next test that will be implemented will be for the other functions of the IC. For example the IC is supposed to begin drawing power from the battery when no other source is present so when the battery is charged the solar panel will be removed from all light sources and the USB will be disconnected. If the battery begins powering the system then we will have verified this function of the IC. From there the battery will be constantly depleted to see if the IC will identify a low battery. If the output voltage of the battery is low then we can confirm independently that it has a low charge and the IC should indicate as such. If it does then this function is confirmed.

\paragraph{Battery Testing}
Battery testing once again must be undertaken in a thorough and unique way. Once we have verified the consistency of the output of the IC we can observe the activity of the battery and identify and unexpected or unplanned elements. Ideally the best way to test this portion of the subsystem would be to charge the battery with a 4.2V output and observe the charge curve of the battery during the charging process. Should the charge curve present itself unlike the example in the power subsection then several issues could be at fault. For instance if the battery is charging faster than predicted and discharging faster as well then potentially the electrolyte mixture in the battery could be compromised. once we've observed a stable and predictable charge curve the next step would be to observe the discharge curve. Just like testing the charge curve one will have to use a voltmeter over the course over a few hours to gain data then graph a discharge curve and compare it to the example graph in the power subsection. Once we've verified that the battery can properly charge and discharge, is getting the proper charge voltage, and is discharging when the original sources are not present or flowing through the IC then we've verified that the battery if functioning properly and can move on further into the system. For ease of use testing points have been added at select points that can be used, these points can be found in the schematic and are the points we will be using to test this subsystem.
\paragraph{Power Rail Testing}
Power rail testing is a lot more straight forward. At this point we have covered the sources, the IC, the battery, and all the connections in between. Should the rest of the system be predictable, stable, and expected now comes the load itself. The power rails comfortable change the voltage from the IC and battery to the voltage needed for every component. The basic and proper way to test this railing would be to using a voltmeter, observe the voltage at the output of the IC and battery to verify that it is indeed correct. Next you would observe that the input voltage of each power rail is the same as the output voltage you just measured, if that's not the case then an element may be present in the connection leading to the difference and the up most care should be taken to dissecting the reason behind the change. Finally for each power rail you would need to ensure that the output voltage of the rail is as predicted. For ease of use testing points have been added at select points that can be used, these points can be found in the schematic and are the points we will be using to test this subsystem.
