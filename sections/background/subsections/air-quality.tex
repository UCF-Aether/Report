\subsection{Effects of Air Pollution}
Pollution is the introduction of substances into the environment that are
harmful to humans and other living organisms
\cite{health-impacts-air-pollution-review}. Pollutants can be solids, liquids,
or gases. Over the years, human activities have had an adverse effect on
the environment by polluting the air we breath, the water we drink, and food that we eat. Climate change is also increasingly creating more air pollution from
wildfires and other chemical reactions involving greenhouse gases. Inhaling
smoke, while potentially short term, can lead to various
health effects. However, air pollutants are more likely to affect those with
predispositions, lower economical statuses, and those living in
urban areas \cite{socioeconomic-disparities-air-pollution-review}.  Human
created air pollution accounts for approximately nine million deaths per year
\cite{health-impacts-air-pollution-review}. Short term exposure to air
pollutants is correlated to Chronic Obstructive Pulmonary Disease (a chronic
inflammation lung disease), cough, shortness of breath, wheezing, asthma, other
respiratory diseases, and increased hospitalization rates. Long term exposure
is correlated to chronic asthma, pulmonary insufficiency (a condition where the
pulmonary valve in your heart allows backflow), cardiovascular diseases, and
cardiovascular mortality. Newborns and children may also develop potentially
life-long cardiovascular, mental, and perinatal disorders. 

There are four 
main sources of air pollution: major sources, indoor sources, mobile sources,
and natural sources \cite{health-impacts-air-pollution-review}. \emph{Major
sources} are sources of pollutants that a created from factories, power
stations, refineries, and other industrial plants. \emph{Indoor sources} are
pollutants that arise from the use of cleaning chemicals, appliances, and other
gas emissions that stay contained inside of industrial buildings. \emph{Mobile
sources} include anything relating to infrastructure, such as cars, trains, and
airplanes. Lastly, \emph{natural sources} arise from naturally occurring
events, which may have to exacerbated due to climate change, such as wildfires
or dust storms. Air pollution can also be absorbed into the ground soil, or
cause acid rain. 


\subsubsection{Types of Air Pollution}
The focus of this project is to be able to detect air pollutants defined by the
U.S.  Environmental Protection Agency (EPA) as being harmful and required to
calculate the Air Quality Index (AQI) to help people identify dangerous air
pollution levels. For communities and cities that suffer from poor air quality,
having more data to help them make changes to their environment, such moving, 
applying pressure to those regulating pollutant producers, or getting alerts of
dangerous events/gas-accumulations, such as wildfires. The AQI is further
discussed in a later section, but it defines a measurable, national standard to
calculate the overall air quality based on five main pollutants: ground level
ozone \ozone, particulate matter (PM2.5 and PM10), carbon monoxide CO, sulfur
dioxide \sdo, and nitrogen dioxide \ndo \cite{technical-aqi}. Short-term
exposure is defined as any exposure continuous over and up-to eight hours. While
there is currently lots of data for short-term exposure to these pollutants, it
is still inconclusive about the full extents of health risks due to long-term
exposure.

\paragraph{Particulate Matter} As defined by the EPA, is
any tiny liquid or solid droplets that can be inhaled and cause serious health
effects, 10 $\mu$m or less in. Coarse particles is defined as particles that are
10 $\mu$m or less. Fine particles are defined as particles being 2.5 $\mu$m or
less. The half-lives of PM$_{2.5}$ and PM$_{10}$ particles are relatively long,
so they last longer in the atmosphere. This can cause particulate matter
pollution to spread over long distances, such as smoke from some extreme
wildfires in California reaching all the way on the east coast. Due to the small
size of PM$_{2.5}$ particles, they can cause more serious health conditions
\cite{health-impacts-air-pollution-review}. Most particles in the atmosphere
are created through chemical reactions with sulfur dioxide and nitrogen oxides,
which are already pollutants created from power plants, factories, and
combustible engines \cite{epa-what-is-pm}.  Particulate matter can be produced
from dust, gas aerosols, and biologics (viruses, microorganisms and allergens)
\cite{health-impacts-air-pollution-review}.

\paragraph{Ground Level Ozone}
According to the EPA, tropospheric (ground-level) ozone \ozone is a highly
reactive gas, and is not naturally forming, unlike stratospheric ozone.
Stratospheric ozone is created through the interaction of solar ultraviolet
(UV) radiation with molecular oxygen (O$_2$). Tropospheric ozone, however, is
formed through photochemical reactions between two classes of air pollutants:
volatile organic compounds (VOCs) and nitrogen oxides (NO$_x$). Anyone that has
ever been to a city on a hot day will noticed the smog and haze, which is the
result of ozone formation. While some VOC and NO$_x$ occur naturally, most
tropospheric ozone is man-made. The main sources of VOCs are: chemical plants,
gasoline pumps, oil-based paints (plastics, petroleum), or recycling facilities.
And, as for NO$_x$, it is mainly found where ever there's combustion, such as
in power plants, industrial furnaces, industrial boilers, and oil-based vehicles
\cite{epa-what-is-ozone}.

\paragraph{Carbon Monoxide}
Carbon monoxide CO is an odorless gas produced through incomplete combustion of
fossil fuels. Carbon monoxide is mainly produced from motor vehicles, or
anything that burns fossil fuels \cite{health-impacts-air-pollution-review}.

\paragraph{Sulfur Dioxide}
Sulfur dioxide \sdo is an colorless gas with a "choking or suffocating odor".
Sulfur dioxide is heavier than air, so on inhalation, it causes the person to
severely cough \cite{pubchem-so2}. \sdo is mainly used to measure for all sulfur
oxides SO$_x$. Oxides are a group of highly reactive gases. But, \sdo can mainly
be measured since other sulfur oxides occur in much smaller quantities
\cite{epa-so2-basics}. It is used to manufacture many different types of
chemicals ranging from household detergents, building materials, such as
flooring, tile, sinks, bathtubs, and drywall, paper pulping, metal, batteries,
and food processing. Sulfur dioxide air pollution mainly comes from burning coal
or oil at power plants or other industrial plants, or copper smelting
\cite{pubchem-so2}. High production of sulfur dioxides can increase the
likelihood of \sdo oxidizing into sulfur trioxide \sto particles. Sulfur oxides
\sox can react in the atmosphere to form poisonous particulate matter
(particles). As previously mentioned, particulate matter, including \sox, can
cause haziness commonly found around cities and busy industrial zones.

\paragraph{Nitrogen Dioxide}
Nitrogen dioxide (\ndo) is a reddish brown gas or yellow-brown liquid. It is a
highly reactive and poisonous gas \cite{pubchem-no2}. Like \sdo, \ndo is a part
of group of highly reactive gases, and is used as an indicator for other forms
of nitrogen oxides. Nitrogen dioxide emissions mainly come from the burning of fuels,
such as from combustion engines and vehicles, power plants, or other industrial
plants \cite{epa-no2-basics}. Like \sox, \ndo and \nox can react to form
particulate matter, which can also be inhaled. Nitrogen oxides are also what
make up for smog and haze during warm seasons in cities and industrial areas.
Nitrogen oxides can also react in the atmosphere with water and other vapors to
form acid rain, which can pollute water supplies and ground soil
\cite{epa-no2-basics}.


\subsubsection{Adverse Effects on Health}

\paragraph{Particulate Matter} (fine particles) can potentially enter the
bloodstream after being inhaled. Long term exposure to PM$_{2.5}$ was found to
be related to cardiovascular diseases and infant mortality.  Particularly,
people with pre-existing respiratory problems will be more susceptible to the
worsened negative effects of PM$_{2.5}$, along with other pollutants. Short-term
exposure to PM has shown to affect people's lungs and heart. Countless
scientific studies have linked particulate matter pollution to the following: premature death
to those with pre-existing lung or heart diseases, nonfatal heart attacks,
irregular heartbeat, aggravated asthma, decreased lung function, airway and lung
inflammation causing coughing or difficulty breathing
\cite{epa-pm-health-effects}. However, there are
studies correlating long-term, high exposure to various air pollutants to
chronic, life-long cardiovascular, neurological, and psychological
complications \cite{health-impacts-air-pollution-review}. Wildfire smoke can
cause long-term, chronic heart and lung disease. Wildfires can put anyone at
risk that are in at-risk areas. Wildfire smoke has also been known to travel
thousands of miles, affecting areas to varying degrees
\cite{epa-how-smoke-affects-health}.

\paragraph{Ozone}
Tropospheric ozone can have many short-term adverse health effects. It is
currently inconclusive whether or not ground-level ozone has any long term
consequences for individuals. However, even short-term exposure has shown to
slightly increases chances of mortality (0.5 to 1.04\%) for the general
population, but especially for those with pre-existing respiratory conditions,
such as asthma, or those over the age of 65. Short-term symptoms include: lung
and airway inflammation, coughing, throat irritation, pain, burning, or
discomfort when breathing. Ozone is able to reach the lower respiratory tract
since it is less water soluble, unlike sulfur dioxide and chlorine gas. Studies
have shown that people exhibit approximately a 50\% drop in forced evacuated
1-second volume (FEV1) (a measure for lung function) for exposures to 400 ppb
over 2 hours, and similarly for a 80 ppb exposure for 5 hours. The result was
severe coughing and an increased rate of asthma attacks
\cite{epa-ozone-health-effects}.

\paragraph{Carbon Monoxide}
Carbon monoxide can, in high indoor levels, cause confusion, dizziness,
unconsciousness, and death. For outdoor levels during short-term exposure, it
can cause chest tightness \cite{epa-co-basics}.

\paragraph{Sulfur Dioxide and Nitrogen Dioxide}
As previously mentioned, \sdo, \ndo, and other oxides (\sox and \nox) are
dangerous, poisonous compounds that come in the form of solid particles and
gases. The EPA has strong control measures for \sdo and \ndo emission producers.
Also, since \sdo and \ndo can oxidize, controlling \sdo and \ndo can also
control its derivates, and in this case, the particulate and gaseous forms of
oxide compounds. \sdo and \ndo have immediate, short-term health effects.  It
can cause choking, since they are both heavier than normal air
\cite{epa-so2-basics}.  However, after removing \sdo from the environment,
people recover from short-term symptoms in about 30 minutes. For people with
asthma, this time is approximately 4 hours
\cite{health-impacts-air-pollution-review}. However, for \ndo, while it may
cause slight pain and lung inflammation, longer exposures can lead to death many
days after when they where exposed \cite{pubchem-no2}.

\begin{table}[H]
\centering\scriptsize
\caption{Unhealthy Exposure Levels \cite{aqi-technical-doc}}
\begin{tabular}{|c|c|c|c|c|c|c|c|c|}
\hline
	PM$_{2.5}$ $\mu$g/m$^3$(24h) & PM$_{10}$ $\mu$g/m$^3$(24h) & CO ppm(8hr) &
	\ozone ppm(1hr) & \sdo ppb(1hr) & \ndo ppb(1hr) \\ 
\hline

	55.5 - 150  & 255 - 354 & 12.5 - 15.4 & 0.165 - 0.204 & 186 - 304 & 361 - 649 \\\hline

\end{tabular}
\label{tab:gasUnhealthyExposure}
\end{table}

% Not sure if this section is needed
% \subsubsection{Climate Change}
