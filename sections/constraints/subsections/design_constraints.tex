\subsection{Design Constraints}
In this section, we discuss all of the realistic design constraints that exist for our project. We will also describe the impact these constraints will have on our design. The constraints that are discussed include economic, time, manufacturability, sustainability, environmental, health, safety, ethical, health, social, and political constraints.

\subsubsection{Economic Constraints}
Cost will be one of the greatest constraints on our design. One of the main ways that this constraint manifests is in the selection of sensors in our design. Since our goal is to measure the concentration of various gases and particulates in the air, the accuracy and sensitivity of the sensors we choose to incorporate in our design is critical. During our research of different types of sensors, we found that sensors can range from being designed for low-cost, hobbyist applications all the way to expensive, high-end sensors that are designed for safety-critical, industrial applications. 

For our design, we have chosen to design a system that can compute the air quality with respect to the Pollutant Standards Index (PSI). Therefore, we only need to select sensors that are sensitive enough to differentiate between the different index categories e.g. from moderate to unhealthy. Any increased sensitivity and accuracy that would come from purchasing more expensive sensors would be unnecessary. This means we should strive to meet this minimum bound for each sensor in our design in order to effectively optimize the cost for our design.

Cost also plays a role in the selection of a microcontroller for our design. The microcontroller will likely be the second most expensive component in our design. The microcontroller is a key component in the design and at the minimum we are looking for one that is packaged with a LoRa module. Some also come packaged with Bluetooth and WiFi transceivers.

\subsubsection{Time Constraints}
Time will play a major role in the development of our project. There will be many deadlines that we will have to meet as we move forward and the complexity of our design will need to be adjusted accordingly. If we decide to add too many features or complexities to our design we may run out of time. While certain design ideas may be better for the functionality of our project, if they take too long to implement then they are not viable and must be discarded in favor of ideas that can be completed within our time frame.

\subsubsection{Manufacturability and Sustainability Constraints}
Manufacturability refers to constraints related to how a design can be produced. If manufacturability is not considered during the design process, the end product may not be possible to build. One of these constraints would be that we should design our circuits to use off the self components. Another manufacturability constraint would be that the PCB we design must be within a certain range of dimensions, as PCB manufacturers will not create a board that is too small or too large.

Sustainability refers to the ability of a design to be long-lasting and to resist degrading over time. Sustainability constraints are important to our design due to the sensor nodes being placed outdoors. We want these sensor nodes to be long lasting, therefore the electronic components of the node should be enclosed in weatherproof housing. In addition, the gas sensors should be shielded from direct rain and sunlight.

\subsubsection{Environmental, Health, and Safety Constraints}
Environmental constraints are constraints that focus on how the design impacts the earth and the usage of the earth's resources. Our design will use a rechargeable battery. In addition, our design will contain solar panels that will be used to recharge the battery. We also must consider the material choice for the enclosure of our design. We currently plan on 3D printing an enclosure, so we must find a material that will not negatively impact the environment. That may include factors, such as not leeching any chemicals into the environment when exposed to water, being a recyclable material, and being a biodegradable material.

Health and safety are constraints that are necessary to keep the design from potentially causing harm to an individual. In our design, one of the main potential safety hazards is electromagnetic radiation. The LoRa IC we use should be certified by the Federal Communications Commission (FCC) in areas related to reducing the of the user to electromagnetic radiation. All electrical contact points and wires shall be properly insulated in order to prevent any injury from electric shock. The housing for the node will be able to be securely mounted to the desired surface to prevent it from falling and potentially injuring someone.

\subsubsection{Ethical, Social, and Political Constraints}
In terms of political constraints, our design shall follow all FCC guidelines when it comes to transmitting radio signals and should be fully certified. The LoRa IC we chose for our design shall be able to transmit within the 900 MHz to 928 MHz frequency band, which is the frequency band designated by the FCC to permit unlicensed transmission. We are not able to determine any ethical or social constraints that apply to our design.